\documentclass[12pt, letterpaper]{article}
\usepackage[utf8]{inputenc}
\usepackage{amsmath}

\title{CS387 - Applied Cryptography}
\author{\'Angel Sola Orbaiceta}
\date{December 2021}

\begin{document}

\begin{titlepage}
  \maketitle
\end{titlepage}

\section{Concepts}

Given a message $m \in M$, where $M$ is the set of all possible messages, and a key $k \in K$, where $K$ is the set of all possible keys, an encryption function $E$ can be defined as:
\[
  E: M \times K \to C
\]
where $c \in C$ is the \emph{ciperthext} (being $C$ the set of all possible ciphertexts).
Conversely, a decryption function $D$ can be defined as:
\[
  D: C \times K \to M
\]

The \textbf{correctness property} states that, for all messages and keys, decrypting the result of encrypting a message must result in the message itself.
Mathematically:

\[
  \forall m,k: D_k (E_k(m)) = m
\]

The \textbf{security property} states that the ciphertext reveals nothing about the key or original message.

\subsection{One-Time Pad}

The one-time pad is based in the XOR ($\oplus$) function.
The XOR function satisfies the property that any value XOR-ed with itself equals zero: $x \oplus x = 0$.
The one-time pad uses this property so that, by using a key that's the same size as the ciphertext, we can do:

\[
  c = m \oplus k\\
\]
\[
  m = c \oplus k
\]

The one-time pad encryption and decryption functions are implemented in the \emph{source/one\_time\_pad.py} file.

\end{document}